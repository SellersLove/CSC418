\RequirePackage[l2tabu,orthodox]{nag}  % warn about common LaTeX pitfalls
\RequirePackage[ascii]{inputenc}  % input is 7-bit ASCII
\RequirePackage{fixltx2e}  % fix LaTeX2e kernel bugs

\documentclass[11pt,twoside]{article}
\usepackage{color}
\usepackage{graphicx}
\graphicspath{ {image/} }
\usepackage{calc}  % arithmetic in length parameters
\usepackage{enumitem}  % more control over list formatting
\usepackage{fancyhdr}  % simpler headers and footers
\usepackage[margin=1in]{geometry}  % page layout
\usepackage{lastpage}  % for last page number
\usepackage{relsize}  % easier font size changes
\usepackage[normalem]{ulem}  % smarter underlining
\usepackage{url}  % verb-like typesetting of URLs
\usepackage{xfrac}  % nicer looking simple fractions for text and math
\usepackage{longtable}
\usepackage{tikz}
\usepackage{array}
\usepackage{tikz-timing}
\usetikzlibrary{arrows, shapes, backgrounds,fit}
\usepackage{tkz-graph}
% Set up fonts.
\usepackage[T1]{fontenc}  % use true 8-bit fonts
\usepackage{slantsc}  % allow slanted small-caps
\usepackage{microtype}  % perform various font optimizations
% Use Palatino-based monospace instead of kpfonts' default.
%\usepackage{newpxtext}
\ttfamily
\DeclareFontShape{T1}{\ttdefault}{m}{scsl}{<->ssub*\ttdefault/m/sc}{}
\DeclareFontShape{T1}{\ttdefault}{b}{scsl}{<->ssub*\ttdefault/b/sc}{}
% "Kepler" fonts.
\usepackage[nott,notextcomp]{kpfonts}
% Use curvier Latin Modern brackets instead of kpfonts' glyphs.
\DeclareSymbolFont{lmsymb}     {OMS}{lmsy}{m}{n}
\DeclareSymbolFont{lmlargesymb}{OMX}{lmex}{m}{n}
\DeclareMathDelimiter{\rbrace}{\mathclose}{lmsymb}{"67}{lmlargesymb}{"09}
\DeclareMathDelimiter{\lbrace}{\mathopen}{lmsymb}{"66}{lmlargesymb}{"08}

% Page layout: stretch text to fill up page.
\addtolength\footskip{.25\headheight}
\flushbottom

% Common list settings.

% Common macros.
\input{macros}
\newcommand*\st{\mathrel{|}}  % "such that" for set extension

% Headings.
\pagestyle{fancy}
\let\headrule\empty
\let\footrule\empty
\lhead{CSC\,418\,H1}
\chead{\large\scshape Assignment \#\,2}
\rhead{\scshape Fall 2015}
\lfoot{\scshape Dept.\@ of Computer Science, University of Toronto,
       St.~George Campus}
\cfoot{}
\rfoot{\scshape page \thepage\space of \pageref{LastPage}}


\begin{document}

\begin{enumerate}[leftmargin=0pt]
%question 1
\item Transformation
	\begin{enumerate}
	\item We know that an affine transformation has the form
				 \[ \left. \begin{bmatrix}
					A & t \\
					0 \ 0 & 1
				\end{bmatrix} \right. \] 
		Now suppose the affine transformation is
				 \[ \left. \begin{bmatrix}
					a &b& x\\
					c &d& y \\
					0 &0 & 1
				\end{bmatrix} \right. \] 
		Then we plug in those points and get 6 equation:
				\[2a+3b+x = 8\]
				\[2c+3d+y = -4\]
				\[a+2b+x = 2\]
				\[c+2d+y = 0\]
				\[3a-b+x = 10\]
				\[3c-d+y = 8\]
		After solving this linear equation we have:
				\[ \left. \begin{bmatrix}
					5.2 &0.8& -4.8\\
					-0.8 &-3.2& 7.2 \\
					0 &0 & 1
				\end{bmatrix} \right. \] 
	\item There are 8 variables for a 2D-Homography; hence, we need 8 points which means 4 maps to 4.\\
		There are 3 variables for a 2D-rigid transform; hence, we need 4 points which means 2 maps to 2.
	\item Both centroid and ortho-center are preserved under affine transformation due the the similarly triangle.\\
	centroid:
	\begin{itemize}[label = {}]
	\item Suppose after an affine transformation a triangle $\Delta ABC$ maps to  $\Delta A'B'C'$. And the centroid $O$ maps to $O'$.
	\item Now, connect  $AO$ and $A'O'$. Suppose line $AO$ intersects $BC$ at $D$ and  $A'O'$ intersects $B'C'$ at $D'$. Clearly  triangle $\Delta ADC$ and $\Delta A'D'C'$ are similarly triangles; hence $D'$ is also the mid-point of $B'C'$.
	\item similarly, we can get $O'$ is the centroid  of $\Delta A'B'C'$.
	\end{itemize}
	ortho-center:
	\begin{itemize}[label = {}]
	\item Suppose after an affine transformation a triangle $\Delta ABC$ maps to  $\Delta A'B'C'$. And the ortho-center $O$ maps to $O'$.
	\item Now, connect  $AO$ and $A'O'$. Suppose line $AO$ intersects $BC$ at $D$ and  $A'O'$ intersects $B'C'$ at $D'$. Clearly  triangle $\Delta ADC$ and $\Delta A'D'C'$ are similarly triangles; hence $A'D'$ is also the perpendicular to $B'C'$.
	\item similarly, we can get $O'$ is the ortho-center  of $\Delta A'B'C'$.
	\end{itemize}
	\end{enumerate}
%question 2
\item Viewing and Projection
 \begin{enumerate}
 \item The function of a real camera would be a complicated function about light, distance etc. \\
 The focal length of the lens is the distance between the lens and the image sensor when the subject is in focus. Smaller focal length would give a broader image view.\\
 Aperture is  an opening through which light travels size of the aperture has a direct impact on the depth of field.
 A smaller aperture will bring all foreground and background objects in focus, while a large aperture will isolate the foreground from the background by making the foreground objects sharp and the background blurry. \item camera position $e(2,1,3)$, look at point $p(-1,2,1)$, and up vector $\vec r(0,1,0)$.
 	Then we have,
			\[\vec s = \frac{p-e}{|| p -e|| }= (\frac{-3}{\sqrt{14}},\frac{1}{\sqrt{14}},\frac{-2}{\sqrt{14}})\]
			\[\vec u = \frac{\vec r \times \vec s}{||\vec r \times \vec s||} = (\frac{-2}{\sqrt{13}},0,\frac{3}{\sqrt{13}})\]
			\[\vec v = \frac{\vec r \times \vec u}{||\vec r \times \vec u||} = (\frac{3}{\sqrt{182}},\frac{13}{\sqrt{182}},\frac{2}{\sqrt{182}})\]
			%\[\vec v = \frac{\vec r \times \vec u}{||\vec r \times \vec u||} = (\frac{-2}{\sqrt{182}},\farc{13}{\sqrt{182}},\frac{2}{\sqrt{182}})\]
	Then we have,
			    \[ M_{cw} = \left. \begin{bmatrix}
					\frac{-2}{\sqrt{13}}&\frac{3}{\sqrt{182}}&\frac{-3}{\sqrt{14}}&2\\
					0&\frac{13}{\sqrt{182}}&\frac{1}{\sqrt{14}}&1\\
					\frac{3}{\sqrt{13}}&\frac{2}{\sqrt{182}}&\frac{-2}{\sqrt{14}}&3\\
					0 &0 & 0& 1
				\end{bmatrix} \right. \] 
	Then 
			 \[ M_{wc} = M^{-1}_{cw} = \left. \begin{bmatrix}
					\frac{-2}{\sqrt{13}}&0&\frac{3}{\sqrt{13}}&\frac{5}{\sqrt{13}}\\
					\frac{3}{\sqrt{182}}&\frac{13}{\sqrt{182}}&\frac{2}{\sqrt{182}}&\frac{22}{\sqrt{182}}\\
					\frac{-3}{\sqrt{14}}&\frac{1}{\sqrt{14}}&\frac{-2}{\sqrt{14}}&\frac{1}{\sqrt{14}}\\
					0 &0 & 0& 1
				\end{bmatrix} \right. \] 
 \item From perspective projection we could get, \[(\frac{dp_x}{p_z},\frac{dp_y}{p_z})\]
 \item Suppose the line function is $l(\lambda) = (a+\lambda b_x, d+\lambda b_y, c+\lambda b_z)$.\\
	 Then the 2D-point of any point point on this line would be,
 		\[ (\frac{d(a+\lambda b_x)}{c+\lambda b_z}, \frac{d(d+\lambda b_y)}{c+\lambda b_z})\]
	As $\lambda$ goes to infinite we get $(\frac{db_x}{b_z},\frac{db_y}{b_z})$, which means the points converge to $(\frac{db_x}{b_z},\frac{db_y}{b_z})$.
  \end{enumerate}
%question 3
\item Surfaces
    \begin{enumerate}
    \item  At any point $p=(x,y,z)$, normal is $(\frac{df}{dx}, \frac{df}{dy} ,\frac{df}{dz})$.
    	Then we have,
    			\[ \frac{df}{dx} = 2x-\frac{2Rx}{\sqrt{x^2+y^2}}\]
    			\[ \frac{df}{dy} = 2y-\frac{2Ry}{\sqrt{x^2+y^2}}\]
    			\[ \frac{df}{dz} = 2z\]
    \item Suppose the function for the tangent plane at point $p$ is $t(q)$ where $q$ is the any point on this tangent plane, then we have,
    			\[t(q)=  \ (q-p)normal_p = 0, \ where \ normal_p  stands \ for \ the \ normal \ vector\ at \ the \ point \ p   \]
    \item Substitute the parametric curve $q(\lambda) = (R\cos\lambda, R\sin\lambda, r)$ into the surface function and we have,
    			\[(R-\sqrt{(R\cos\lambda)^2+(R\cos\lambda)^2})^2 + r^2-r^2 \]
			\[= (R-\sqrt{R^2})^2 + 0 \]
			\[= 0 \]
	Hence, parametric curve is on the surface.
  \item the tangent vector for  $q(\lambda) $ is $(\frac{dx}{d\lambda}, \frac{dx}{d\lambda}, z )$; therefore we have the tangent vector,
  			\[(-R\sin\lambda, R\cos\lambda, r ) \]
	for any point on curve $q(\lambda) $.
  \item First we know that the normal vector for any point at plane is,
  			\[(2R-\frac{2R^2(\cos(\lambda))}{R}, 2R-\frac{2R^2(\sin(\lambda))}{R}, 2r)\] 
	 \ \ \ \ \ \ \ which is,
	 		\[(0, 0, 2r)\] 
	   And from $b$ we know that for any point $q$, if its tangent vector $(x,y,z)$ lie on the tangent plane, we will have,
	   		\[ 0(x-R\cos\lambda)+0(y-R\sin\lambda) +2r(z-r) = 0\]
	\ \ \ \ \ \ \ which is,
	 		\[rz-z^2 = 0\] 
	Now from $(d)$ we have that the tangent vector for any point at the curve $p(\lambda)$ is $(-R\sin\lambda, R\cos\lambda, r ) $.\\
	Then we substitute this into $rz-z^2 = 0$, and we have,
			\[r^2-r^2=0\]
			Hence, the this tangent vector of $q(\lambda)$ does lie the the tangent plane.	
    \end{enumerate}
\item
 \begin{enumerate}
	\item It is possible to exclude things, for instance  it is possible to  exclude  $d$, $d$  might be blocked by $c$ and it is pointing aways from the general camera view.
	\item  
                   \[ \begin{tikzpicture}[scale=0.2]
                    \tikzstyle{every node}+=[inner sep=0pt]
                    \draw [black] (38.2,-5.5) circle (3);
                    \draw (38.2,-5.5) node {$h$};
                    \draw [black] (29.9,-14.3) circle (3);
                    \draw (29.9,-14.3) node {$c$};
                    \draw [black] (46.7,-13.5) circle (3);
                    \draw (46.7,-13.5) node {$a$};
                    \draw [black] (22.3,-23) circle (3);
                    \draw (22.3,-23) node {$d$};
                    \draw [black] (14.3,-32.9) circle (3);
                    \draw (14.3,-32.9) node {$e$};
                    \draw [black] (37.6,-23) circle (3);
                    \draw (37.6,-23) node {$g$};
                    \draw [black] (44.5,-31.5) circle (3);
                    \draw (44.5,-31.5) node {$b$};
                    \draw [black] (34.8,-39.6) circle (3);
                    \draw (34.8,-39.6) node {$l$};
                    \draw [black] (55.4,-22.1) circle (3);
                    \draw (55.4,-22.1) node {$f$};
                    \draw [black] (36.14,-7.68) -- (31.96,-12.12);
                    \fill [black] (31.96,-12.12) -- (32.87,-11.88) -- (32.14,-11.19);
                    \draw [black] (27.93,-16.56) -- (24.27,-20.74);
                    \fill [black] (24.27,-20.74) -- (25.18,-20.47) -- (24.42,-19.81);
                    \draw [black] (31.89,-16.55) -- (35.61,-20.75);
                    \fill [black] (35.61,-20.75) -- (35.46,-19.82) -- (34.71,-20.49);
                    \draw [black] (20.41,-25.33) -- (16.19,-30.57);
                    \fill [black] (16.19,-30.57) -- (17.08,-30.26) -- (16.3,-29.63);
                    \draw [black] (39.49,-25.33) -- (42.61,-29.17);
                    \fill [black] (42.61,-29.17) -- (42.49,-28.23) -- (41.72,-28.86);
                    \draw [black] (42.2,-33.42) -- (37.1,-37.68);
                    \fill [black] (37.1,-37.68) -- (38.04,-37.55) -- (37.4,-36.78);
                    \draw [black] (48.83,-15.61) -- (53.27,-19.99);
                    \fill [black] (53.27,-19.99) -- (53.05,-19.07) -- (52.35,-19.78);
                    \draw [black] (40.38,-7.56) -- (44.52,-11.44);
                    \fill [black] (44.52,-11.44) -- (44.28,-10.53) -- (43.59,-11.26);
                    \end{tikzpicture} \]
          \item We based on the rule that if $eye$ is outside a face $i$ 
          	\begin{itemize}
		\item draw everything inside $i$
		\item draw $i$
		\item draw everything out side $i$
		\end{itemize}
		 if $eye$ is inside a face $i$ 
          	\begin{itemize}
		\item draw everything outside $i$
		\item draw $i$
		\item draw everything inside $i$
		\end{itemize}
		Hence, we would get, 
          $[\ a,\ f,\ h,\ d,\ e,\ c,\ b,\ l, \ g,]$
         \end{enumerate}
         	
\end{enumerate}

\end{document}